\documentclass[a4j,10pt,oneside,openany]{jsbook}
%
\usepackage{amsmath,amssymb}
\usepackage{bm}
\usepackage{graphicx}
\usepackage{ascmac}
\usepackage{makeidx}
%
\makeindex
%
\newcommand{\diff}{\mathrm{d}}  %微分記号
\newcommand{\divergence}{\mathrm{div}\,}  %ダイバージェンス
\newcommand{\grad}{\mathrm{grad}\,}  %グラディエント
\newcommand{\rot}{\mathrm{rot}\,}  %ローテーション
%
\setlength{\textwidth}{\fullwidth}
\setlength{\textheight}{44\baselineskip}
\addtolength{\textheight}{\topskip}
\setlength{\voffset}{-0.6in}
%
\title{{\Huge \textbf{Mixed Realityにおける注視物体の透明度と選択の行いやすさの関係についての研究
}}\\ }
\author{桑津賢士}
\date{\today}
\begin{document}
%
%
\maketitle
\frontmatter
\tableofcontents
%
%
\mainmatter

\chapter{...}
\begin{abstract}
...
...
\end{abstract}

\section{...}
...
...

\begin{thebibliography}{20}
 \bibitem{...}...
 ...
\end{thebibliography}

\newpage
\printindex
%
%
\end{document}